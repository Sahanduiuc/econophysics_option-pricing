\chapter{Methodes}

\section{Figuren}
Een voorbeerbeeld is Figuur~\ref{Landslide}
\begin{figure}[h] % [h] bepaalt de plaats waar de figuur komt in de tekst
    \centering % figuur komt in het midden terecht
    \includegraphics[width=0.8\textwidth]{Figuren/landslide3.jpg}
    \caption{Een landslide.}
    \label{Landslide}
\end{figure}

\newpage

\section{Tabellen}
Een voorbeerbeeld is Tabel~\ref{Tabel1}
\begin{table}[h]
    \centering
    \begin{tabular}{|c|c|}
        \hline
       \bf Model  & \bf Nauwkeurigheid  \\ \hline
        regressie & 90\%                \\ \hline
        random forests & 95\%           \\ \hline
    \end{tabular}
    \caption{Een willekeurige tabel.}
    \label{Tabel1}
\end{table}

\newpage

\section{Vergelijkingen}
Vergelijkingen en/of functies kan je ofwel in de tekst zelf schrijven door de functie tussen twee \$-tekens te zetten (zogenaamde \textit{mathmode}, bijvoorbeeld $Y_i=\frac{1}{x}$), of door een aparte vergelijking als `float' in de tekst toe te voegen. Bijvoorbeeld Vergelijking~\ref{Eq1}.
\begin{equation}\label{Eq1}
    y=\frac{1}{x}
\end{equation}

\begin{equation}
    y=\int_{a}^{b} x^2 dx
\end{equation}

\begin{equation}
    y=\sum_{i=1}^{n} x_i^2
\end{equation}

Je kan ook alle vergelijkingen aligneren:

\begin{align}
    y &=\frac{1}{x} \\
    y &=\int_{a}^{b} x^2 dx \\
    y &=\sum_{i=1}^{n} x_i^2
\end{align}